%\section{SPOJ MAXMATCH -- Maximum Self-Matching}

\begin{frame}[fragile]{Problema}

You're given a string $s$ consisting of letters `\texttt{a}', `\texttt{b}' and `\texttt{c}'.

The matching function $m_s(i)$ is defined as the number of matching characters of $s$ and its
$i$-shift. In other words, $m_s(i)$ is the number of characters that are matched when you align the
0-th character of $s$ with the $i$-th character of its copy.

You are asked to compute the maximum of $m_s(i)$ for all $i$ $(1\leq i\leq |s|)$. To make it a bit
harder, you should also output all the optimal $i$'s in increasing order.

\end{frame}

\begin{frame}[fragile]{Entrada e saída}

\textbf{Input}

The first and only line of input contains the string $s$ ($2\leq |s| \leq 10^5$).

\vspace{0.2in}

\textbf{Output}

The first line of output contains the maximal $m_s(i)$ over all $i$.

The second line of output contains all the $i$'s for which $m_s(i)$ reaches maximum.

\end{frame}

\begin{frame}[fragile]{Exemplo de entradas e saídas}

\begin{minipage}[t]{0.45\textwidth}
\textbf{Sample Input}
\begin{verbatim}
caccacaa
\end{verbatim}
\end{minipage}
\begin{minipage}[t]{0.5\textwidth}
\textbf{Sample Output}
\begin{verbatim}
4
3
\end{verbatim}
\end{minipage}
\end{frame}

\begin{frame}[fragile]{Solução $O(N^2)$}

    \begin{itemize}
        \item A função $m_s(i)$ corresponde à distância de Hamming entre a string $s$ e a 
            substring $b_i = s[i..(N - 1)]$, com $i = 1, 2, \ldots, N$

        \item Esta distância de Hamming entre as strings $s$ e $t$ é dada por
        \[
            D(s, t) = \sum_{i = 0}^m (1 - \delta_{s[i]}(t[i])),
        \]
        onde $m = \min(|s|, |t|)$ e $\delta_j(j) = 1$ e $\delta_j(k) = 0$, se $j\neq k$

        \item Assim, $D$ tem complexidade $O(N)$

        \item Uma solução que computa $D(s, b_i)$ para todos os valores de $i$ tem complexidade
            $O(N^2)$, e leva ao veredito TLE
    \end{itemize}

\end{frame}

\begin{frame}[fragile]{Solução $O(N\log N)$}

    \begin{itemize}
        \item Considere as strings $t_k$ tais que $t_k[i] = \delta_k(s[i])$

        \item Na string dada no exemplo, $t_a =$ ``\texttt{01001011}'', $t_b =$
            ``\texttt{00000000}'' e $t_c =$ ``\texttt{10110100}''

        \item A ideia é computar $m_s(i)$ como a soma das funções $m^k_i(s)$, com $k =$ 
            `\texttt{a}', `\texttt{b}' e `\texttt{c}', onde $m^k_i(s)$ é calculada a partir da
            string $t_k$

        \item Usando esta representação binária das strings, o cálculo da distância de Hamming 
            corresponde ao produto escalar entre a string $t_k$ e a substring $b_i$

        \item Estes produtos escalares surgem na multiplicação dos polinômios correspondentes a
            strings $t_k$ e a reversa da string $b_i$

    \end{itemize}

\end{frame}

\begin{frame}[fragile]{Solução $O(N\log N)$}

    \begin{itemize}
        \item Assim, os valores de $m^k_i(s)$ para cada $i$ podem ser computados todos de uma
            só vez, por meio da multiplicação de polinômios, em $O(N\log N)$

        \item Atente que, devido à multiplicação polinomial, o valor de $m^k_i(s)$ será o 
            coeficiente do monômio de grau $i + N$, onde $N$ é o tamando da string $t_k$

        \item Embora $m^k_i(N) = 0$, é preciso considerá-lo na composição final da resposta,
            uma vez que $0$ pode ser o valor máximo obtido, e neste caso o índice $N$ também
            deve ser listado

        \item Repetido o processo para cada valor de $k$, o problema pode ser resolvido em
            $O(N\log N)$
    \end{itemize}

\end{frame}

\begin{frame}[fragile]{Solução $O(N\log N)$}
    \inputsnippet{cpp}{1}{19}{codes/MAXMATCH.cpp}
\end{frame}

\begin{frame}[fragile]{Solução $O(N\log N)$}
    \inputsnippet{cpp}{21}{39}{codes/MAXMATCH.cpp}
\end{frame}

\begin{frame}[fragile]{Solução $O(N\log N)$}
    \inputsnippet{cpp}{41}{60}{codes/MAXMATCH.cpp}
\end{frame}

\begin{frame}[fragile]{Solução $O(N\log N)$}
    \inputsnippet{cpp}{62}{81}{codes/MAXMATCH.cpp}
\end{frame}

\begin{frame}[fragile]{Solução $O(N\log N)$}
    \inputsnippet{cpp}{83}{96}{codes/MAXMATCH.cpp}
\end{frame}

\begin{frame}[fragile]{Solução $O(N\log N)$}
    \inputsnippet{cpp}{98}{116}{codes/MAXMATCH.cpp}
\end{frame}

\begin{frame}[fragile]{Solução $O(N\log N)$}
    \inputsnippet{cpp}{118}{126}{codes/MAXMATCH.cpp}
\end{frame}

\begin{frame}[fragile]{Bônus: redução da constante por meio de bijeção}
    \begin{itemize}
        \item É possível reduzir a constante da solução por meio do uso de uma bijeção

        \item Seja $T$ o texto principal e $P$ o padrão a ser identificado em substrings de tamanho $P$ de $T$

        \item Define, para cada caractere $\alpha$ do alfabeto $\Sigma$, o vetor binário $v_{T_\alpha}$, onde
            $v_{T_\alpha}(i) = 1$ se $T[i] = \alpha$, ou zero, caso contrário

        \item Defina, de forma semelhante, o vetor $v_{P_\alpha}$

        \item A solução anterior realizada a operação
        \[
            \mathrm{IFFT}(\mathrm{FFT}(v_{T_\alpha})\otimes\mathrm{FFT}(v_{P_\alpha}))
        \]
        $|\Sigma|$ vezes, onde $\otimes$ é a multiplicação termo-a-termo

    \end{itemize}

\end{frame}

\begin{frame}[fragile]{Bônus: redução da constante por meio de bijeção}

    \begin{itemize}
        \item É possível reduzir o número de operações para $|\Sigma|/2$ por meio de uma bijeção entre os caracteres de $\Sigma$ e números complexos

        \item Neste mapeamento, cada letra deve estar associada a um ângulo do círculo trigonométrico

        \item Para que as operações preservem as propriedades desejadas, os ângulo escolhidos devem ser as raízes $|\Sigma|$-ésimas da unidade

        \item Por exemplo, para $N = 3$, os caracteres devem ser mapeados aos ângulos $0, \frac{2\pi}{3}$ e $\frac{4\pi}{3}$

        \item As raízes $|\Sigma|$-ésimas da unidade são dadas por
        \[
            w_k = \cos\left(\frac{2\pi k}{|\Sigma|}\right) + i\sin\left(\frac{2\pi k}{|\Sigma|}\right), \ \ \ \ k = 0, 1, \ldots, |\Sigma| - 1
        \]

    \end{itemize}

\end{frame}

\begin{frame}[fragile]{Bônus: redução da constante por meio de bijeção}
    \begin{itemize}
        \item Além da relação com as raízes da unidade, o mapeamento deve atender o seguinte critério: no produto termo-a-termo das transformadas de 
            $v_{T_\alpha}$ e $v_{P_\alpha}$, o produto entre caracteres iguais deve resultar em 1; entre caracteres diferentes, deve resultar em um número
            complexo com parte real igual a $\cos(\frac{2\pi}{|\Sigma|})$

        \item Isto pode ser alcançado mantendo-se os índices de 0 a $|\Sigma| - 1$ para o alfabeto do texto $T$, e usar os complementares $j = |\Sigma| - i$ para
            os índices do alfabeto para o padrão $P$

        \item Por exemplo, para $\Sigma = \{\ a, b, c\ \}$, os índices para $T$ seriam $a = 0, b = 1, c = 2$, e para $P$ teríamos $a = 0, b = 2, c = 1$

        \item Observe que
        \begin{align*}
            a\cdot a = w_0\cdot w_0 = 1\\
            b\cdot b = w_1\cdot w_2 = 1\\
            c\cdot c = w_2\cdot w_1 = 1\\
        \end{align*}
    \end{itemize}
\end{frame}

\begin{frame}[fragile]{Bônus: redução da constante por meio de bijeção}
    \begin{itemize}
        \item Ainda no caso $N = 3$, dado o número de comparações $t$ de $P$ sobre $T$ e a parte real da convolução para cada índice $p$, vale que
        \[
            D(T, P, t) = \mathrm{round}\left(\frac{t + 2\mathrm{Re}(v(p))}{|\Sigma|}\right),
        \]
        onde $D(T, P, t)$ é a distância de Hamming entre $T$ e $P$ numa substring de tamanho $t$

        \item Usando o caso de teste do problema, no cenário abaixo
        \begin{mdframed}
        \begin{minted}{c}
            caccacaa
               caccacca
        \end{minted}
        \end{mdframed}
         temos $T =  caccacaa, P = caccacaa, |\Sigma| = 3$.

        \item São 4 posições corretas e 1 incorreta, de modo que a soma dos $\mathrm{Re}(v(p))$ seria igual a $4\cdot 1 - 1\cdot \frac{1}{2} = 3.5$, e como $t = 5$, vale que
        $\mathrm{round}((5 + 2\cdot 3.5)/3) = 4$
 
    \end{itemize}
\end{frame}

\begin{frame}[fragile]{Solução $O(N\log N)$}
    \inputsnippet{cpp}{1}{18}{codes/MAXMATCH-BIJECTIVE.cpp}
\end{frame}

\begin{frame}[fragile]{Solução $O(N\log N)$}
    \inputsnippet{cpp}{20}{34}{codes/MAXMATCH-BIJECTIVE.cpp}
\end{frame}

\begin{frame}[fragile]{Solução $O(N\log N)$}
    \inputsnippet{cpp}{34}{52}{codes/MAXMATCH-BIJECTIVE.cpp}
\end{frame}

\begin{frame}[fragile]{Solução $O(N\log N)$}
    \inputsnippet{cpp}{54}{72}{codes/MAXMATCH-BIJECTIVE.cpp}
\end{frame}

\begin{frame}[fragile]{Solução $O(N\log N)$}
    \inputsnippet{cpp}{74}{93}{codes/MAXMATCH-BIJECTIVE.cpp}
\end{frame}

\begin{frame}[fragile]{Referência}

    \textbf{SCHOENMEYR, Tor}, \textbf{ZHANG, David Yu}. \href{https://www.sciencedirect.com/science/article/pii/S0196677405000180}{FFT-based algorithms for the string matching with mismatches problem}. Journal of Algorithms, Volume 57, Issue 2, November 2004, pg. 130-139.
\end{frame}
